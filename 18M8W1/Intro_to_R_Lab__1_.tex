\documentclass[]{article}
\usepackage{lmodern}
\usepackage{amssymb,amsmath}
\usepackage{ifxetex,ifluatex}
\usepackage{fixltx2e} % provides \textsubscript
\ifnum 0\ifxetex 1\fi\ifluatex 1\fi=0 % if pdftex
  \usepackage[T1]{fontenc}
  \usepackage[utf8]{inputenc}
\else % if luatex or xelatex
  \ifxetex
    \usepackage{mathspec}
  \else
    \usepackage{fontspec}
  \fi
  \defaultfontfeatures{Ligatures=TeX,Scale=MatchLowercase}
\fi
% use upquote if available, for straight quotes in verbatim environments
\IfFileExists{upquote.sty}{\usepackage{upquote}}{}
% use microtype if available
\IfFileExists{microtype.sty}{%
\usepackage{microtype}
\UseMicrotypeSet[protrusion]{basicmath} % disable protrusion for tt fonts
}{}
\usepackage[margin=1in]{geometry}
\usepackage{hyperref}
\hypersetup{unicode=true,
            pdftitle={Intro to R Lab},
            pdfauthor={Ayodele Odubela},
            pdfborder={0 0 0},
            breaklinks=true}
\urlstyle{same}  % don't use monospace font for urls
\usepackage{color}
\usepackage{fancyvrb}
\newcommand{\VerbBar}{|}
\newcommand{\VERB}{\Verb[commandchars=\\\{\}]}
\DefineVerbatimEnvironment{Highlighting}{Verbatim}{commandchars=\\\{\}}
% Add ',fontsize=\small' for more characters per line
\usepackage{framed}
\definecolor{shadecolor}{RGB}{248,248,248}
\newenvironment{Shaded}{\begin{snugshade}}{\end{snugshade}}
\newcommand{\KeywordTok}[1]{\textcolor[rgb]{0.13,0.29,0.53}{\textbf{#1}}}
\newcommand{\DataTypeTok}[1]{\textcolor[rgb]{0.13,0.29,0.53}{#1}}
\newcommand{\DecValTok}[1]{\textcolor[rgb]{0.00,0.00,0.81}{#1}}
\newcommand{\BaseNTok}[1]{\textcolor[rgb]{0.00,0.00,0.81}{#1}}
\newcommand{\FloatTok}[1]{\textcolor[rgb]{0.00,0.00,0.81}{#1}}
\newcommand{\ConstantTok}[1]{\textcolor[rgb]{0.00,0.00,0.00}{#1}}
\newcommand{\CharTok}[1]{\textcolor[rgb]{0.31,0.60,0.02}{#1}}
\newcommand{\SpecialCharTok}[1]{\textcolor[rgb]{0.00,0.00,0.00}{#1}}
\newcommand{\StringTok}[1]{\textcolor[rgb]{0.31,0.60,0.02}{#1}}
\newcommand{\VerbatimStringTok}[1]{\textcolor[rgb]{0.31,0.60,0.02}{#1}}
\newcommand{\SpecialStringTok}[1]{\textcolor[rgb]{0.31,0.60,0.02}{#1}}
\newcommand{\ImportTok}[1]{#1}
\newcommand{\CommentTok}[1]{\textcolor[rgb]{0.56,0.35,0.01}{\textit{#1}}}
\newcommand{\DocumentationTok}[1]{\textcolor[rgb]{0.56,0.35,0.01}{\textbf{\textit{#1}}}}
\newcommand{\AnnotationTok}[1]{\textcolor[rgb]{0.56,0.35,0.01}{\textbf{\textit{#1}}}}
\newcommand{\CommentVarTok}[1]{\textcolor[rgb]{0.56,0.35,0.01}{\textbf{\textit{#1}}}}
\newcommand{\OtherTok}[1]{\textcolor[rgb]{0.56,0.35,0.01}{#1}}
\newcommand{\FunctionTok}[1]{\textcolor[rgb]{0.00,0.00,0.00}{#1}}
\newcommand{\VariableTok}[1]{\textcolor[rgb]{0.00,0.00,0.00}{#1}}
\newcommand{\ControlFlowTok}[1]{\textcolor[rgb]{0.13,0.29,0.53}{\textbf{#1}}}
\newcommand{\OperatorTok}[1]{\textcolor[rgb]{0.81,0.36,0.00}{\textbf{#1}}}
\newcommand{\BuiltInTok}[1]{#1}
\newcommand{\ExtensionTok}[1]{#1}
\newcommand{\PreprocessorTok}[1]{\textcolor[rgb]{0.56,0.35,0.01}{\textit{#1}}}
\newcommand{\AttributeTok}[1]{\textcolor[rgb]{0.77,0.63,0.00}{#1}}
\newcommand{\RegionMarkerTok}[1]{#1}
\newcommand{\InformationTok}[1]{\textcolor[rgb]{0.56,0.35,0.01}{\textbf{\textit{#1}}}}
\newcommand{\WarningTok}[1]{\textcolor[rgb]{0.56,0.35,0.01}{\textbf{\textit{#1}}}}
\newcommand{\AlertTok}[1]{\textcolor[rgb]{0.94,0.16,0.16}{#1}}
\newcommand{\ErrorTok}[1]{\textcolor[rgb]{0.64,0.00,0.00}{\textbf{#1}}}
\newcommand{\NormalTok}[1]{#1}
\usepackage{graphicx,grffile}
\makeatletter
\def\maxwidth{\ifdim\Gin@nat@width>\linewidth\linewidth\else\Gin@nat@width\fi}
\def\maxheight{\ifdim\Gin@nat@height>\textheight\textheight\else\Gin@nat@height\fi}
\makeatother
% Scale images if necessary, so that they will not overflow the page
% margins by default, and it is still possible to overwrite the defaults
% using explicit options in \includegraphics[width, height, ...]{}
\setkeys{Gin}{width=\maxwidth,height=\maxheight,keepaspectratio}
\IfFileExists{parskip.sty}{%
\usepackage{parskip}
}{% else
\setlength{\parindent}{0pt}
\setlength{\parskip}{6pt plus 2pt minus 1pt}
}
\setlength{\emergencystretch}{3em}  % prevent overfull lines
\providecommand{\tightlist}{%
  \setlength{\itemsep}{0pt}\setlength{\parskip}{0pt}}
\setcounter{secnumdepth}{0}
% Redefines (sub)paragraphs to behave more like sections
\ifx\paragraph\undefined\else
\let\oldparagraph\paragraph
\renewcommand{\paragraph}[1]{\oldparagraph{#1}\mbox{}}
\fi
\ifx\subparagraph\undefined\else
\let\oldsubparagraph\subparagraph
\renewcommand{\subparagraph}[1]{\oldsubparagraph{#1}\mbox{}}
\fi

%%% Use protect on footnotes to avoid problems with footnotes in titles
\let\rmarkdownfootnote\footnote%
\def\footnote{\protect\rmarkdownfootnote}

%%% Change title format to be more compact
\usepackage{titling}

% Create subtitle command for use in maketitle
\newcommand{\subtitle}[1]{
  \posttitle{
    \begin{center}\large#1\end{center}
    }
}

\setlength{\droptitle}{-2em}
  \title{Intro to R Lab}
  \pretitle{\vspace{\droptitle}\centering\huge}
  \posttitle{\par}
  \author{Ayodele Odubela}
  \preauthor{\centering\large\emph}
  \postauthor{\par}
  \predate{\centering\large\emph}
  \postdate{\par}
  \date{5/11/2018}


\begin{document}
\maketitle

Introduction to R Lab:

2.3.1- Basic Commands

In tis section the basic comamands of finding the length of vector as
well as learning about (?) and adding data to a matrix. Here I also
explore creating a matrix of normal random variables (mean of 0 and sd
of 1) using the rnorm() function. I also set seeds for the random number
functions in order to get reproducible random numbers. I also explored
summary statistics for the variable y.

\begin{Shaded}
\begin{Highlighting}[]
\NormalTok{x <-}\StringTok{ }\KeywordTok{c}\NormalTok{(}\DecValTok{1}\NormalTok{,}\DecValTok{2}\NormalTok{,}\DecValTok{3}\NormalTok{,}\DecValTok{5}\NormalTok{)}
\NormalTok{x}
\end{Highlighting}
\end{Shaded}

\begin{verbatim}
## [1] 1 2 3 5
\end{verbatim}

\begin{Shaded}
\begin{Highlighting}[]
\KeywordTok{length}\NormalTok{(x)}
\end{Highlighting}
\end{Shaded}

\begin{verbatim}
## [1] 4
\end{verbatim}

\begin{Shaded}
\begin{Highlighting}[]
\NormalTok{y <-}\StringTok{ }\KeywordTok{c}\NormalTok{(}\DecValTok{7}\NormalTok{,}\DecValTok{8}\NormalTok{,}\DecValTok{9}\NormalTok{)}
\NormalTok{x }\OperatorTok{+}\StringTok{ }\NormalTok{y}
\end{Highlighting}
\end{Shaded}

\begin{verbatim}
## Warning in x + y: longer object length is not a multiple of shorter object
## length
\end{verbatim}

\begin{verbatim}
## [1]  8 10 12 12
\end{verbatim}

\begin{Shaded}
\begin{Highlighting}[]
\KeywordTok{ls}\NormalTok{()}
\end{Highlighting}
\end{Shaded}

\begin{verbatim}
## [1] "x" "y"
\end{verbatim}

\begin{Shaded}
\begin{Highlighting}[]
\KeywordTok{rm}\NormalTok{(x,y)}
\KeywordTok{rm}\NormalTok{(}\DataTypeTok{list=}\KeywordTok{ls}\NormalTok{())}
\NormalTok{?matrix}
\end{Highlighting}
\end{Shaded}

\begin{verbatim}
## starting httpd help server ... done
\end{verbatim}

\begin{Shaded}
\begin{Highlighting}[]
\NormalTok{x =}\StringTok{ }\KeywordTok{matrix}\NormalTok{(}\DataTypeTok{data=}\KeywordTok{c}\NormalTok{(}\DecValTok{1}\NormalTok{,}\DecValTok{2}\NormalTok{,}\DecValTok{3}\NormalTok{,}\DecValTok{4}\NormalTok{), }\DataTypeTok{nrow=} \DecValTok{2}\NormalTok{, }\DataTypeTok{ncol =} \DecValTok{2}\NormalTok{)}
\NormalTok{x}
\end{Highlighting}
\end{Shaded}

\begin{verbatim}
##      [,1] [,2]
## [1,]    1    3
## [2,]    2    4
\end{verbatim}

\begin{Shaded}
\begin{Highlighting}[]
\NormalTok{x =}\StringTok{ }\KeywordTok{matrix}\NormalTok{(}\KeywordTok{c}\NormalTok{(}\DecValTok{1}\NormalTok{,}\DecValTok{2}\NormalTok{,}\DecValTok{3}\NormalTok{,}\DecValTok{4}\NormalTok{), }\DecValTok{2}\NormalTok{,}\DecValTok{2}\NormalTok{)}
\NormalTok{x}
\end{Highlighting}
\end{Shaded}

\begin{verbatim}
##      [,1] [,2]
## [1,]    1    3
## [2,]    2    4
\end{verbatim}

\begin{Shaded}
\begin{Highlighting}[]
\KeywordTok{matrix}\NormalTok{(}\KeywordTok{c}\NormalTok{(}\DecValTok{1}\NormalTok{,}\DecValTok{2}\NormalTok{,}\DecValTok{3}\NormalTok{,}\DecValTok{4}\NormalTok{), }\DecValTok{2}\NormalTok{,}\DecValTok{2}\NormalTok{, }\DataTypeTok{byrow =} \OtherTok{TRUE}\NormalTok{)}
\end{Highlighting}
\end{Shaded}

\begin{verbatim}
##      [,1] [,2]
## [1,]    1    2
## [2,]    3    4
\end{verbatim}

\begin{Shaded}
\begin{Highlighting}[]
\KeywordTok{sqrt}\NormalTok{(x)}
\end{Highlighting}
\end{Shaded}

\begin{verbatim}
##          [,1]     [,2]
## [1,] 1.000000 1.732051
## [2,] 1.414214 2.000000
\end{verbatim}

\begin{Shaded}
\begin{Highlighting}[]
\NormalTok{x}\OperatorTok{^}\DecValTok{2}
\end{Highlighting}
\end{Shaded}

\begin{verbatim}
##      [,1] [,2]
## [1,]    1    9
## [2,]    4   16
\end{verbatim}

\begin{Shaded}
\begin{Highlighting}[]
\NormalTok{x=}\StringTok{ }\KeywordTok{rnorm}\NormalTok{(}\DecValTok{50}\NormalTok{)}
\NormalTok{y =}\StringTok{ }\NormalTok{x}\OperatorTok{+}\StringTok{ }\KeywordTok{rnorm}\NormalTok{(}\DecValTok{50}\NormalTok{, }\DataTypeTok{mean =} \DecValTok{50}\NormalTok{, }\DataTypeTok{sd=}\NormalTok{ .}\DecValTok{1}\NormalTok{)}
\KeywordTok{cor}\NormalTok{(x,y)}
\end{Highlighting}
\end{Shaded}

\begin{verbatim}
## [1] 0.9920701
\end{verbatim}

\begin{Shaded}
\begin{Highlighting}[]
\KeywordTok{set.seed}\NormalTok{(}\DecValTok{1303}\NormalTok{)}
\KeywordTok{rnorm}\NormalTok{(}\DecValTok{50}\NormalTok{)}
\end{Highlighting}
\end{Shaded}

\begin{verbatim}
##  [1] -1.1439763145  1.3421293656  2.1853904757  0.5363925179  0.0631929665
##  [6]  0.5022344825 -0.0004167247  0.5658198405 -0.5725226890 -1.1102250073
## [11] -0.0486871234 -0.6956562176  0.8289174803  0.2066528551 -0.2356745091
## [16] -0.5563104914 -0.3647543571  0.8623550343 -0.6307715354  0.3136021252
## [21] -0.9314953177  0.8238676185  0.5233707021  0.7069214120  0.4202043256
## [26] -0.2690521547 -1.5103172999 -0.6902124766 -0.1434719524 -1.0135274099
## [31]  1.5732737361  0.0127465055  0.8726470499  0.4220661905 -0.0188157917
## [36]  2.6157489689 -0.6931401748 -0.2663217810 -0.7206364412  1.3677342065
## [41]  0.2640073322  0.6321868074 -1.3306509858  0.0268888182  1.0406363208
## [46]  1.3120237985 -0.0300020767 -0.2500257125  0.0234144857  1.6598706557
\end{verbatim}

\begin{Shaded}
\begin{Highlighting}[]
\KeywordTok{set.seed}\NormalTok{(}\DecValTok{3}\NormalTok{)}
\NormalTok{y =}\StringTok{ }\KeywordTok{rnorm}\NormalTok{(}\DecValTok{100}\NormalTok{)}
\KeywordTok{mean}\NormalTok{(y)}
\end{Highlighting}
\end{Shaded}

\begin{verbatim}
## [1] 0.01103557
\end{verbatim}

\begin{Shaded}
\begin{Highlighting}[]
\KeywordTok{var}\NormalTok{(y)}
\end{Highlighting}
\end{Shaded}

\begin{verbatim}
## [1] 0.7328675
\end{verbatim}

\begin{Shaded}
\begin{Highlighting}[]
\KeywordTok{sqrt}\NormalTok{(y)}
\end{Highlighting}
\end{Shaded}

\begin{verbatim}
## Warning in sqrt(y): NaNs produced
\end{verbatim}

\begin{verbatim}
##   [1]       NaN       NaN 0.5087123       NaN 0.4424735 0.1735625 0.2922631
##   [8] 1.0566978       NaN 1.1257747       NaN       NaN       NaN 0.5026454
##  [15] 0.3899304       NaN       NaN       NaN 1.1064871 0.4470029       NaN
##  [22]       NaN       NaN       NaN       NaN       NaN 1.0773188 1.0060155
##  [29]       NaN       NaN 0.9490125 0.9229141 0.8530622 0.8581970       NaN
##  [36] 0.8399497 1.1403324 0.1955812       NaN 0.8909328 0.8868522       NaN
##  [43] 1.3034128       NaN 0.5902861       NaN       NaN 1.0634214       NaN
##  [50]       NaN 0.8525485       NaN 0.5168028       NaN       NaN       NaN
##  [57]       NaN 1.1671088 0.9578396       NaN 0.7573098 0.9582256 0.5062482
##  [64] 0.5932677 1.0836685       NaN       NaN 0.9772987       NaN 0.4315060
##  [71]       NaN 0.6834452 1.0120265 0.5170672 0.4814832 0.8646343 1.1032083
##  [78] 0.6191594       NaN       NaN 1.3173971       NaN 0.8298434 1.1065288
##  [85] 0.8912330       NaN 0.4681353       NaN 0.6631442       NaN       NaN
##  [92]       NaN       NaN 1.0266190       NaN       NaN       NaN 0.7354647
##  [99] 0.9652124       NaN
\end{verbatim}

\begin{Shaded}
\begin{Highlighting}[]
\KeywordTok{sd}\NormalTok{(y)}
\end{Highlighting}
\end{Shaded}

\begin{verbatim}
## [1] 0.8560768
\end{verbatim}

2.3.2 - Graphics

In this section I used the plot() function to display my matrix of
random data. I used parameters xlab, ylab, and main to create labels for
the plot as well as a title. I created a pdf and jpeg of my plot. I also
used the seq() function to create a vector sequence. I implemented
contour() to fine-tune my plot and persp() to make at 3D plot.

\begin{Shaded}
\begin{Highlighting}[]
\NormalTok{x =}\StringTok{ }\KeywordTok{rnorm}\NormalTok{(}\DecValTok{100}\NormalTok{)}
\NormalTok{y =}\StringTok{ }\KeywordTok{rnorm}\NormalTok{(}\DecValTok{100}\NormalTok{)}
\KeywordTok{plot}\NormalTok{(x,y)}
\end{Highlighting}
\end{Shaded}

\includegraphics{Intro_to_R_Lab__1__files/figure-latex/graphics-1.pdf}

\begin{Shaded}
\begin{Highlighting}[]
\KeywordTok{plot}\NormalTok{(x, y, }\DataTypeTok{xlab=} \StringTok{"this is the x axis"}\NormalTok{, }\DataTypeTok{ylab =} \StringTok{"this is the y axis"}\NormalTok{, }\DataTypeTok{main =} \StringTok{"Plot of x and y"}\NormalTok{)}
\end{Highlighting}
\end{Shaded}

\includegraphics{Intro_to_R_Lab__1__files/figure-latex/graphics-2.pdf}

\begin{Shaded}
\begin{Highlighting}[]
\KeywordTok{pdf}\NormalTok{(}\StringTok{"Figure 1.pdf"}\NormalTok{)}
\KeywordTok{jpeg}\NormalTok{(}\StringTok{"Figure1j.jpeg"}\NormalTok{)}
\KeywordTok{plot}\NormalTok{(x,y,}\DataTypeTok{col =} \StringTok{"green"}\NormalTok{)}
\KeywordTok{dev.off}\NormalTok{()}
\end{Highlighting}
\end{Shaded}

\begin{verbatim}
## pdf 
##   2
\end{verbatim}

\begin{Shaded}
\begin{Highlighting}[]
\NormalTok{x =}\KeywordTok{seq}\NormalTok{(}\DecValTok{1}\NormalTok{, }\DecValTok{10}\NormalTok{)}
\NormalTok{x}
\end{Highlighting}
\end{Shaded}

\begin{verbatim}
##  [1]  1  2  3  4  5  6  7  8  9 10
\end{verbatim}

\begin{Shaded}
\begin{Highlighting}[]
\NormalTok{x =}\StringTok{ }\DecValTok{1}\OperatorTok{:}\DecValTok{10}
\NormalTok{x}
\end{Highlighting}
\end{Shaded}

\begin{verbatim}
##  [1]  1  2  3  4  5  6  7  8  9 10
\end{verbatim}

\begin{Shaded}
\begin{Highlighting}[]
\NormalTok{y =}\StringTok{ }\NormalTok{x}
\NormalTok{f =}\StringTok{ }\KeywordTok{outer}\NormalTok{(x,y, }\ControlFlowTok{function}\NormalTok{(x,y)}\KeywordTok{cos}\NormalTok{(y)}\OperatorTok{/}\NormalTok{(}\DecValTok{1}\OperatorTok{+}\NormalTok{x}\OperatorTok{^}\DecValTok{2}\NormalTok{))}
\KeywordTok{contour}\NormalTok{(x,y,f)}
\KeywordTok{contour}\NormalTok{(x,y,f, }\DataTypeTok{nlevels=}\DecValTok{45}\NormalTok{, }\DataTypeTok{add =}\NormalTok{ T)}
\end{Highlighting}
\end{Shaded}

\includegraphics{Intro_to_R_Lab__1__files/figure-latex/graphics-3.pdf}

\begin{Shaded}
\begin{Highlighting}[]
\NormalTok{fa =}\StringTok{ }\NormalTok{(f}\OperatorTok{-}\KeywordTok{t}\NormalTok{(f))}\OperatorTok{/}\DecValTok{2}
\KeywordTok{contour}\NormalTok{(x,y,fa, }\DataTypeTok{nlevels =} \DecValTok{15}\NormalTok{)}
\end{Highlighting}
\end{Shaded}

\includegraphics{Intro_to_R_Lab__1__files/figure-latex/graphics-4.pdf}

\begin{Shaded}
\begin{Highlighting}[]
\KeywordTok{image}\NormalTok{(x, y, fa)}
\end{Highlighting}
\end{Shaded}

\includegraphics{Intro_to_R_Lab__1__files/figure-latex/graphics-5.pdf}

\begin{Shaded}
\begin{Highlighting}[]
\KeywordTok{persp}\NormalTok{(x, y, fa)}
\end{Highlighting}
\end{Shaded}

\includegraphics{Intro_to_R_Lab__1__files/figure-latex/graphics-6.pdf}

\begin{Shaded}
\begin{Highlighting}[]
\KeywordTok{persp}\NormalTok{(x, y, fa, }\DataTypeTok{theta =} \DecValTok{30}\NormalTok{)}
\end{Highlighting}
\end{Shaded}

\includegraphics{Intro_to_R_Lab__1__files/figure-latex/graphics-7.pdf}

\begin{Shaded}
\begin{Highlighting}[]
\KeywordTok{persp}\NormalTok{(x, y, fa, }\DataTypeTok{theta =} \DecValTok{30}\NormalTok{, }\DataTypeTok{phi =} \DecValTok{30}\NormalTok{)}
\end{Highlighting}
\end{Shaded}

\includegraphics{Intro_to_R_Lab__1__files/figure-latex/graphics-8.pdf}

\begin{Shaded}
\begin{Highlighting}[]
\KeywordTok{persp}\NormalTok{(x, y, fa, }\DataTypeTok{theta =} \DecValTok{30}\NormalTok{, }\DataTypeTok{phi =} \DecValTok{70}\NormalTok{)}
\end{Highlighting}
\end{Shaded}

\includegraphics{Intro_to_R_Lab__1__files/figure-latex/graphics-9.pdf}

\begin{Shaded}
\begin{Highlighting}[]
\KeywordTok{persp}\NormalTok{(x, y, fa, }\DataTypeTok{theta =} \DecValTok{30}\NormalTok{, }\DataTypeTok{phi =} \DecValTok{40}\NormalTok{)}
\end{Highlighting}
\end{Shaded}

\includegraphics{Intro_to_R_Lab__1__files/figure-latex/graphics-10.pdf}

2.3.3- Indexing Data

In this section I learned how to slice and index matrices with brackets,
colons, and commas. This section on row/column selection I found
important when it comes to selecting the RIGHT data.

\begin{Shaded}
\begin{Highlighting}[]
\NormalTok{A =}\StringTok{ }\KeywordTok{matrix}\NormalTok{(}\DecValTok{1}\OperatorTok{:}\DecValTok{16}\NormalTok{, }\DecValTok{4}\NormalTok{, }\DecValTok{4}\NormalTok{)}
\NormalTok{A}
\end{Highlighting}
\end{Shaded}

\begin{verbatim}
##      [,1] [,2] [,3] [,4]
## [1,]    1    5    9   13
## [2,]    2    6   10   14
## [3,]    3    7   11   15
## [4,]    4    8   12   16
\end{verbatim}

\begin{Shaded}
\begin{Highlighting}[]
\NormalTok{A[}\DecValTok{2}\NormalTok{,}\DecValTok{3}\NormalTok{]}
\end{Highlighting}
\end{Shaded}

\begin{verbatim}
## [1] 10
\end{verbatim}

\begin{Shaded}
\begin{Highlighting}[]
\NormalTok{A[}\KeywordTok{c}\NormalTok{(}\DecValTok{1}\NormalTok{,}\DecValTok{3}\NormalTok{), }\KeywordTok{c}\NormalTok{(}\DecValTok{2}\NormalTok{,}\DecValTok{4}\NormalTok{)]}
\end{Highlighting}
\end{Shaded}

\begin{verbatim}
##      [,1] [,2]
## [1,]    5   13
## [2,]    7   15
\end{verbatim}

\begin{Shaded}
\begin{Highlighting}[]
\NormalTok{A[}\DecValTok{1}\OperatorTok{:}\DecValTok{3}\NormalTok{, }\DecValTok{2}\OperatorTok{:}\DecValTok{4}\NormalTok{]}
\end{Highlighting}
\end{Shaded}

\begin{verbatim}
##      [,1] [,2] [,3]
## [1,]    5    9   13
## [2,]    6   10   14
## [3,]    7   11   15
\end{verbatim}

\begin{Shaded}
\begin{Highlighting}[]
\NormalTok{A[}\DecValTok{1}\OperatorTok{:}\DecValTok{2}\NormalTok{,]}
\end{Highlighting}
\end{Shaded}

\begin{verbatim}
##      [,1] [,2] [,3] [,4]
## [1,]    1    5    9   13
## [2,]    2    6   10   14
\end{verbatim}

\begin{Shaded}
\begin{Highlighting}[]
\NormalTok{A[,}\DecValTok{1}\OperatorTok{:}\DecValTok{2}\NormalTok{]}
\end{Highlighting}
\end{Shaded}

\begin{verbatim}
##      [,1] [,2]
## [1,]    1    5
## [2,]    2    6
## [3,]    3    7
## [4,]    4    8
\end{verbatim}

\begin{Shaded}
\begin{Highlighting}[]
\NormalTok{A[}\DecValTok{1}\NormalTok{,]}
\end{Highlighting}
\end{Shaded}

\begin{verbatim}
## [1]  1  5  9 13
\end{verbatim}

\begin{Shaded}
\begin{Highlighting}[]
\NormalTok{A[}\OperatorTok{-}\KeywordTok{c}\NormalTok{(}\DecValTok{1}\NormalTok{,}\DecValTok{3}\NormalTok{) , ]}
\end{Highlighting}
\end{Shaded}

\begin{verbatim}
##      [,1] [,2] [,3] [,4]
## [1,]    2    6   10   14
## [2,]    4    8   12   16
\end{verbatim}

\begin{Shaded}
\begin{Highlighting}[]
\KeywordTok{dim}\NormalTok{(A)}
\end{Highlighting}
\end{Shaded}

\begin{verbatim}
## [1] 4 4
\end{verbatim}

2.3.4 - Loading Data

I am very familliar with loading data using functions like read.table()
and read.csv() as well as their parameters for header and now to handle
NA values. I used the mtcars data here which is preloaded into RStudio.

2.3.5 - Graphical and Numerical summaries

Here I create a scatterplot of the cylinder and MPG data from the mtcars
dataset. This plot shows clearly that the more cylinders a car has, the
lower its MPG. I also created a boxplot using the same datapoints to
better highlight the outliers that were recieving unusually high or low
MPG. I created a bar chart and fine tuned its output graphics with
additional parameters including x and y axis labels. The hist() function
easily created a histogram and pairs() created a scatterplot matrix. I
used the summary() function to show general summary of the dataset. The
most important stats are included like min, max, median, and mean.

\begin{Shaded}
\begin{Highlighting}[]
\NormalTok{auto =}\StringTok{ }\NormalTok{mtcars}
\CommentTok{#fix(auto)}
\KeywordTok{dim}\NormalTok{(auto)}
\end{Highlighting}
\end{Shaded}

\begin{verbatim}
## [1] 32 11
\end{verbatim}

\begin{Shaded}
\begin{Highlighting}[]
\NormalTok{auto =}\StringTok{ }\KeywordTok{na.omit}\NormalTok{(auto)}
\KeywordTok{names}\NormalTok{(auto)}
\end{Highlighting}
\end{Shaded}

\begin{verbatim}
##  [1] "mpg"  "cyl"  "disp" "hp"   "drat" "wt"   "qsec" "vs"   "am"   "gear"
## [11] "carb"
\end{verbatim}

\begin{Shaded}
\begin{Highlighting}[]
\KeywordTok{plot}\NormalTok{(auto}\OperatorTok{$}\NormalTok{cyl, auto}\OperatorTok{$}\NormalTok{mpg)}
\end{Highlighting}
\end{Shaded}

\includegraphics{Intro_to_R_Lab__1__files/figure-latex/unnamed-chunk-2-1.pdf}

\begin{Shaded}
\begin{Highlighting}[]
\KeywordTok{attach}\NormalTok{(auto)}
\KeywordTok{plot}\NormalTok{(cyl, mpg)}
\end{Highlighting}
\end{Shaded}

\includegraphics{Intro_to_R_Lab__1__files/figure-latex/unnamed-chunk-2-2.pdf}

\begin{Shaded}
\begin{Highlighting}[]
\NormalTok{cyl=}\StringTok{ }\KeywordTok{as.factor}\NormalTok{(cyl)}
\KeywordTok{plot}\NormalTok{(cyl, mpg)}
\end{Highlighting}
\end{Shaded}

\includegraphics{Intro_to_R_Lab__1__files/figure-latex/unnamed-chunk-2-3.pdf}

\begin{Shaded}
\begin{Highlighting}[]
\KeywordTok{plot}\NormalTok{(cyl, }\DataTypeTok{col=}\StringTok{"red"}\NormalTok{)}
\end{Highlighting}
\end{Shaded}

\includegraphics{Intro_to_R_Lab__1__files/figure-latex/unnamed-chunk-2-4.pdf}

\begin{Shaded}
\begin{Highlighting}[]
\KeywordTok{plot}\NormalTok{(cyl, }\DataTypeTok{col =} \StringTok{"red"}\NormalTok{, }\DataTypeTok{varwidth=}\NormalTok{T)}
\end{Highlighting}
\end{Shaded}

\begin{verbatim}
## Warning in plot.window(xlim, ylim, log = log, ...): "varwidth" is not a
## graphical parameter
\end{verbatim}

\begin{verbatim}
## Warning in axis(if (horiz) 2 else 1, at = at.l, labels = names.arg, lty =
## axis.lty, : "varwidth" is not a graphical parameter
\end{verbatim}

\begin{verbatim}
## Warning in title(main = main, sub = sub, xlab = xlab, ylab = ylab, ...):
## "varwidth" is not a graphical parameter
\end{verbatim}

\begin{verbatim}
## Warning in axis(if (horiz) 1 else 2, cex.axis = cex.axis, ...): "varwidth"
## is not a graphical parameter
\end{verbatim}

\includegraphics{Intro_to_R_Lab__1__files/figure-latex/unnamed-chunk-2-5.pdf}

\begin{Shaded}
\begin{Highlighting}[]
\KeywordTok{plot}\NormalTok{(cyl, }\DataTypeTok{col =} \StringTok{"red"}\NormalTok{, }\DataTypeTok{varwidth=}\NormalTok{T, }\DataTypeTok{horizontal =}\NormalTok{ T)}
\end{Highlighting}
\end{Shaded}

\begin{verbatim}
## Warning in plot.window(xlim, ylim, log = log, ...): "varwidth" is not a
## graphical parameter
\end{verbatim}

\begin{verbatim}
## Warning in plot.window(xlim, ylim, log = log, ...): "horizontal" is not a
## graphical parameter
\end{verbatim}

\begin{verbatim}
## Warning in axis(if (horiz) 2 else 1, at = at.l, labels = names.arg, lty =
## axis.lty, : "varwidth" is not a graphical parameter
\end{verbatim}

\begin{verbatim}
## Warning in axis(if (horiz) 2 else 1, at = at.l, labels = names.arg, lty =
## axis.lty, : "horizontal" is not a graphical parameter
\end{verbatim}

\begin{verbatim}
## Warning in title(main = main, sub = sub, xlab = xlab, ylab = ylab, ...):
## "varwidth" is not a graphical parameter
\end{verbatim}

\begin{verbatim}
## Warning in title(main = main, sub = sub, xlab = xlab, ylab = ylab, ...):
## "horizontal" is not a graphical parameter
\end{verbatim}

\begin{verbatim}
## Warning in axis(if (horiz) 1 else 2, cex.axis = cex.axis, ...): "varwidth"
## is not a graphical parameter
\end{verbatim}

\begin{verbatim}
## Warning in axis(if (horiz) 1 else 2, cex.axis = cex.axis, ...):
## "horizontal" is not a graphical parameter
\end{verbatim}

\includegraphics{Intro_to_R_Lab__1__files/figure-latex/unnamed-chunk-2-6.pdf}

\begin{Shaded}
\begin{Highlighting}[]
\KeywordTok{plot}\NormalTok{(cyl, }\DataTypeTok{col =} \StringTok{"red"}\NormalTok{, }\DataTypeTok{varwidth=}\NormalTok{T, }\DataTypeTok{horizontal =}\NormalTok{ T, }\DataTypeTok{xlab =} \StringTok{"cylinders"}\NormalTok{, }\DataTypeTok{ylab =} \StringTok{"mpg"}\NormalTok{)}
\end{Highlighting}
\end{Shaded}

\begin{verbatim}
## Warning in plot.window(xlim, ylim, log = log, ...): "varwidth" is not a
## graphical parameter
\end{verbatim}

\begin{verbatim}
## Warning in plot.window(xlim, ylim, log = log, ...): "horizontal" is not a
## graphical parameter
\end{verbatim}

\begin{verbatim}
## Warning in axis(if (horiz) 2 else 1, at = at.l, labels = names.arg, lty =
## axis.lty, : "varwidth" is not a graphical parameter
\end{verbatim}

\begin{verbatim}
## Warning in axis(if (horiz) 2 else 1, at = at.l, labels = names.arg, lty =
## axis.lty, : "horizontal" is not a graphical parameter
\end{verbatim}

\begin{verbatim}
## Warning in title(main = main, sub = sub, xlab = xlab, ylab = ylab, ...):
## "varwidth" is not a graphical parameter
\end{verbatim}

\begin{verbatim}
## Warning in title(main = main, sub = sub, xlab = xlab, ylab = ylab, ...):
## "horizontal" is not a graphical parameter
\end{verbatim}

\begin{verbatim}
## Warning in axis(if (horiz) 1 else 2, cex.axis = cex.axis, ...): "varwidth"
## is not a graphical parameter
\end{verbatim}

\begin{verbatim}
## Warning in axis(if (horiz) 1 else 2, cex.axis = cex.axis, ...):
## "horizontal" is not a graphical parameter
\end{verbatim}

\includegraphics{Intro_to_R_Lab__1__files/figure-latex/unnamed-chunk-2-7.pdf}

\begin{Shaded}
\begin{Highlighting}[]
\KeywordTok{hist}\NormalTok{(mpg, }\DataTypeTok{col =} \DecValTok{2}\NormalTok{)}
\end{Highlighting}
\end{Shaded}

\includegraphics{Intro_to_R_Lab__1__files/figure-latex/unnamed-chunk-2-8.pdf}

\begin{Shaded}
\begin{Highlighting}[]
\KeywordTok{hist}\NormalTok{(mpg, }\DataTypeTok{col =} \DecValTok{2}\NormalTok{, }\DataTypeTok{breaks =} \DecValTok{15}\NormalTok{)}
\end{Highlighting}
\end{Shaded}

\includegraphics{Intro_to_R_Lab__1__files/figure-latex/unnamed-chunk-2-9.pdf}

\begin{Shaded}
\begin{Highlighting}[]
\KeywordTok{pairs}\NormalTok{(auto)}
\end{Highlighting}
\end{Shaded}

\includegraphics{Intro_to_R_Lab__1__files/figure-latex/unnamed-chunk-2-10.pdf}

\begin{Shaded}
\begin{Highlighting}[]
\KeywordTok{pairs}\NormalTok{(}\OperatorTok{~}\NormalTok{mpg }\OperatorTok{+}\StringTok{ }\NormalTok{disp }\OperatorTok{+}\StringTok{ }\NormalTok{hp }\OperatorTok{+}\StringTok{ }\NormalTok{wt, auto)}
\end{Highlighting}
\end{Shaded}

\includegraphics{Intro_to_R_Lab__1__files/figure-latex/unnamed-chunk-2-11.pdf}

\begin{Shaded}
\begin{Highlighting}[]
\KeywordTok{plot}\NormalTok{(hp, mpg)}
\KeywordTok{identify}\NormalTok{(hp, mpg)}
\end{Highlighting}
\end{Shaded}

\includegraphics{Intro_to_R_Lab__1__files/figure-latex/unnamed-chunk-2-12.pdf}

\begin{verbatim}
## integer(0)
\end{verbatim}

\begin{Shaded}
\begin{Highlighting}[]
\KeywordTok{summary}\NormalTok{(auto)}
\end{Highlighting}
\end{Shaded}

\begin{verbatim}
##       mpg             cyl             disp             hp       
##  Min.   :10.40   Min.   :4.000   Min.   : 71.1   Min.   : 52.0  
##  1st Qu.:15.43   1st Qu.:4.000   1st Qu.:120.8   1st Qu.: 96.5  
##  Median :19.20   Median :6.000   Median :196.3   Median :123.0  
##  Mean   :20.09   Mean   :6.188   Mean   :230.7   Mean   :146.7  
##  3rd Qu.:22.80   3rd Qu.:8.000   3rd Qu.:326.0   3rd Qu.:180.0  
##  Max.   :33.90   Max.   :8.000   Max.   :472.0   Max.   :335.0  
##       drat             wt             qsec             vs        
##  Min.   :2.760   Min.   :1.513   Min.   :14.50   Min.   :0.0000  
##  1st Qu.:3.080   1st Qu.:2.581   1st Qu.:16.89   1st Qu.:0.0000  
##  Median :3.695   Median :3.325   Median :17.71   Median :0.0000  
##  Mean   :3.597   Mean   :3.217   Mean   :17.85   Mean   :0.4375  
##  3rd Qu.:3.920   3rd Qu.:3.610   3rd Qu.:18.90   3rd Qu.:1.0000  
##  Max.   :4.930   Max.   :5.424   Max.   :22.90   Max.   :1.0000  
##        am              gear            carb      
##  Min.   :0.0000   Min.   :3.000   Min.   :1.000  
##  1st Qu.:0.0000   1st Qu.:3.000   1st Qu.:2.000  
##  Median :0.0000   Median :4.000   Median :2.000  
##  Mean   :0.4062   Mean   :3.688   Mean   :2.812  
##  3rd Qu.:1.0000   3rd Qu.:4.000   3rd Qu.:4.000  
##  Max.   :1.0000   Max.   :5.000   Max.   :8.000
\end{verbatim}

\begin{Shaded}
\begin{Highlighting}[]
\KeywordTok{summary}\NormalTok{(mpg)}
\end{Highlighting}
\end{Shaded}

\begin{verbatim}
##    Min. 1st Qu.  Median    Mean 3rd Qu.    Max. 
##   10.40   15.43   19.20   20.09   22.80   33.90
\end{verbatim}


\end{document}
